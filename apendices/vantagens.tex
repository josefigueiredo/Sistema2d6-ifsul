\chapter{\label{apendiceVantagensDesvantagens}Lista Vantagens e Desvantagens}

atributos podem ser especificos, qualquer atributo ou especial. Especial significa que não é preciso atributo

\section*{Vantagens}
As vantagens listadas possuem Nome, atributo relacionado (sempre entre parêntesis) e uma explicação. Logo abaixo uma tabela com:
\begin{itemize}
	\item \textbf{Custo} - informando o custo para compra da vantagem. O custo poderá ser único ou variado, nos casos de vantagem com níveis diferentes.
	\item \textbf{Ativação} - informa quantos Pontos de Energia são necessários para ativar a vantagem. O traço (-) é usado para informar que a vantagem não usa PEs, ou seja, está sempre ativada.
	\item \textbf{Descrição} - informa o que muda quando com a vantagem.
\end{itemize}


\begin{small}

\textbf{Absorção de Dano (CON)}: O personagem absorve instantaneamente uma quantidade do dano sofrido. \\
\begin{tabu}{|X[c]|X[c]|X[3]|} \tabucline-
	\textbf{Custo} 	& \textbf{Ativação}	&	\textbf{Descrição} \\ \tabucline-		
	1 ponto	& 2 PEs &	Absorve 3 de dano \\ \tabucline-
	2 ponto	& 2 PEs &	Absorve 6 de dano \\ \tabucline-
	3 ponto	& 2 PEs &	Absorve 9 de dano \\ \tabucline-
	4 ponto	& 2 PEs &	Absorve 12 de dano \\ \tabucline-
	5 ponto	& 2 PEs &	Absorve 15 de dano \\ \tabucline-
\end{tabu}


\textbf{Adaptabilidade Cultural (INT)}: O personagem consegue se adaptar a ambientes culturais ou alienígenas totalmente diferentes do que está acostumado. 

\begin{tabu}{|X[c]|X[c]|X[3]|} \tabucline-
		\textbf{Custo} 	& \textbf{Ativação}	&	\textbf{Descrição} \\ \tabucline-		
		1 ponto	& - &	Ganha +2 em testes de Carisma. \\ \tabucline-
\end{tabu}
	

\textbf{Aliados (CAR)}: Personagem tem aliados (PdMs), que estarão à disposição do personagem. A quantidade máxima de aliados é 3. \\
\begin{tabu}{|X[c]|X[c]|X[3]|} \tabucline-
		\textbf{Custo} 	& \textbf{Ativação}	&	\textbf{Descrição} \\ \tabucline-		
		1 ponto	& - &	Tem 1 aliado \\ \tabucline-
		2 ponto	& - &	Tem 2 aliados \\ \tabucline-
		3 ponto	& - &	Tem 3 aliados \\ \tabucline-
	\end{tabu}


\textbf{Anfíbio (CON)}: Personagem tem capacidades de anfíbio. Não consome PEs.

	\begin{tabu}{|X[c]|X[c]|X[3]|} \tabucline-
	\textbf{Custo} 			& \textbf{Ativação}		&	\textbf{Descrição} \\ \tabucline-
		2 pontos			&		- 				&	Pode respirar embaixo da água. \\ \tabucline-
	\end{tabu}


\textbf{Aparência encantadora (CAR)}: Personagem possui uma beleza incomum. 

	\begin{tabu}{|X[c]|X[c]|X[3 c]|} \tabucline-
		\textbf{Custo} 	& \textbf{Ativação}	&	\textbf{Descrição} \\ \tabucline-
		1 ponto					&	- 							&	Ganha +2 em testes sociais. \\ \tabucline-
		2 ponto					&	-							&	Ganha +4 em testes sociais. \\ \tabucline-
		3 ponto					&	-							&	Ganha +6 em testes sociais. \\ \tabucline-
	\end{tabu}


\textbf{Ataque Especial (Qualquer atributos)}: O personagem possui um ataque especial que garante dano extra.

	\begin{tabu}{|X[c]|X[c]|X[3]|} \tabucline-
	\textbf{Custo} 	& \textbf{Ativação}	&	\textbf{Descrição} \\ \tabucline-
	1 ponto		&	1 PEs. 			& Ganha +1d6 de dano extra em um alvo. \\ \tabucline-
	2 pontos	&	2 PEs 			& Ganha +2d6 de dano extra em um alvo ou +1d6 em dois alvos. \\ \tabucline-
	3 pontos	&	3 PEs 			& Ganha +3d6 de dano extra em um alvo, ou +2d6 em dois alvos ou +1d6 em três alvos.\\ \tabucline-
	4 pontos	&	4 PEs 			& Ganha +4d6 de dano extra em um alvo, ou +3d6 em dois alvos ou +2d6 em três alvos. \\ \tabucline-
	5 pontos	&	5 PEs 			& Ganha +5d6 de dano extra em um alvo, ou +4d6 de dano extra em dois alvos ou +3d6 em três alvos. \\ \tabucline-
	\end{tabu}


\textbf{Aumento de velocidade (DES)}: Personagem tem um corpo super-veloz.

	\begin{tabu}{|X[c]|X[c]|X[3]|} \tabucline-
		\textbf{Custo} 		& \textbf{Ativação}		&	\textbf{Descrição} \\ \tabucline-
		3 pontos			&	4 PEs				&	Ganha DES+10, durante 10 minutos. \\ \hline
	\end{tabu}



\textbf{Aumento da densidade (PODER)}: Personagem pode aumentar a densidade do seu corpo temporariamente, ganhando Redução de Dano extra.

	\begin{tabu}{|X[c]|X[c]|X[3]|} \tabucline-
	\textbf{Custo} 	& \textbf{Ativação}	&	\textbf{Descrição} \\ \tabucline-
	3 pontos				&	3 PEs 						&	Ganha RD = 15, durante 10 minutos. \\ \tabucline-
	\end{tabu}



\textbf{Braços Cibernéticos (DES ou FOR)}: Personagem possui implantes cibernéticos nos braços, que aumentam a força e destreza. Os implantes precisam de  recarga (com eletricidade, carvão, plutônio, etc) e manutenção periódicas (definir a periodicidade com o Mestre).

	\begin{tabu}{|X[c]|X[c]|X[3]|} \tabucline-
	\textbf{Custo} 	& \textbf{Ativação}	&	\textbf{Descrição} \\ \tabucline-
	\multirow{2}{*}{4 pontos}	& 	2 PEs			& Ganha +4 em FOR e DES por 10 minutos . \\ \tabucline{2-3}
								& 	4 PEs			& Causa uma descarga elétrica ao tocar o alvo com 2d6 de dano \\ \tabucline-
	\end{tabu}

\textbf{Camaleão (CAR)}: De alguma forma, o personagem possui a capacidade de mudar sua forma física, clonando a aparência de um humanoide (pessoa ou criatura com forma humana). O personagem pode gastar mais PEs para aumentar o tempo da clonagem até o máximo de 60 minutos. Atributos, perícias, vantagens e desvantagens não são clonados.\\
\begin{tabu}{|X[c]|X[c]|X[3]|} \tabucline-
	\textbf{Custo} 	& \textbf{Ativação}	&	\textbf{Descrição} \\ \tabucline-
	3	& 	3 PEs			& Clona a aparência um humanoide que o personagem já tenha visto por 10 minutos. Ganha +4 em Carisma em testes sociais quando estiver usando a aparência de outra pessoa. \\ \tabucline-
	5	& 	4 PEs			& Clona um humanoide famoso na história, que o personagem já tenha visto por 10 minutos. Ganha +6 em testes sociais quando estiver usando a aparência do famoso. Definir com o Mestre quem são os famosos conhecidos.\\ \tabucline-
\end{tabu}


\textbf{Carga Extra (FOR)}: Personagem possui habilidade para carregar mais carga. Carga Extra não aumenta dano.

	\begin{tabu}{|X[c]|X[c]|X[3]|} \tabucline-
		\textbf{Custo} 	& \textbf{Ativação}	&	\textbf{Descrição} \\ \tabucline-
		2	& 	-			& Ganha FOR+3 para cálculo de carga que poder carregar. \\ \tabucline-
	\end{tabu}


\textbf{Corajoso (INT)}: Personagem é muito corajoso.

	\begin{tabu}{|X[c]|X[c]|X[3]|} \tabucline-
		\textbf{Custo} 	& \textbf{Ativação}	&	\textbf{Descrição} \\ \tabucline-
		1	& 	-			& Ganha +2 em testes de medo ou insanidade. \\ \tabucline-
		2	& 	-			& Ganha +4 em testes de medo ou insanidade. \\ \tabucline-
		3	& 	-			& Ganha +6 em testes de medo ou insanidade. \\ \tabucline-
		4	& 	-			& Ganha +8 em testes de medo ou insanidade. \\ \tabucline-
		5	& 	-			& Ganha +10 em testes de medo ou insanidade. \\ \tabucline-
	\end{tabu}


\textbf{Corpo de Água (CON)}: Por algum motivo o corpo do personagem é feito de água e tem forma física maleável. Se chegar a 0 PVs, personagem morre e o corpo se dissolve. 

	\begin{tabu}{|X[c]|X[c]|X[3]|} \tabucline-
		\textbf{Custo} 	& \textbf{Ativação}	&	\textbf{Descrição} \\ \tabucline-
		\multirow{2}{*}{5}	& 	-	& Ganha RD=5 e tem forma física completamente maleável. \\ \tabucline{2-3}
							& 	-	& Usando DES pode tentar envolver alvo com o corpo, causando 2d6+3 de dano por afogamento a cada rodada. \\ \tabucline-
	\end{tabu}


\textbf{Corpo de Ar (CON)}: Por algum motivo o corpo do personagem é feito de ar e tem forma física maleável. Se chegar a 0 PVs, personagem morre com o corpo dissipando no ar. 

	\begin{tabu}{|X[c]|X[c]|X[3]|} \tabucline-
	\textbf{Custo} 	& \textbf{Ativação}	&	\textbf{Descrição} \\ \tabucline-
	\multirow{2}{*}{5}	& 	-	& Recebe 1d6 de dano extra por ataques de energia (fogo, eletricidade, etc). Ataques tipo ventania podem ser feitos a 10m de distância do alvo e causam dano de Força.\\ \tabucline{2-3}
						& 	2	& Pode voar até 60 km/h por 10 minutos. \\ \tabucline-
	\end{tabu}


\textbf{Corpo de Fogo (CON)}: Por algum motivo o corpo do personagem é feito de fogo e tem forma física maleável. Se chegar a 0 PVs, personagem morre com o corpo desfaz em fumaça.

	\begin{tabu}{|X[c]|X[c]|X[3]|} \tabucline-
		\textbf{Custo} 	& \textbf{Ativação}	&	\textbf{Descrição} \\ \tabucline-
		\multirow{3}{*}{5}	& 	-	& Recebe 1d6 de dano extra por ataques com água ou gelo. \\ \tabucline{2-3}
		& 	2	& Pode voar a um Deslocamento 36 km/h. \\ \tabucline{2-3}
		&	3	&	Solta um raio de fogo a até 10 metros que causa 2d6+3 de dano.
		\\ \tabucline-
	\end{tabu}


fogo
O corpo do personagem é feito de fogo. Forma maleável. 
Seu toque causa 2d6 extras de dano de fogo. 
Sofre 1d6 de dano extra sempre que sofrer ataque com água ou gelo. Não sofre penalidades por ferimentos, mas pode ter membros amputados. 

Gastando 2 PE pode esticar um dos membros a até 3 metros. Se chegar a 0 PVs morre, com o corpo dissolvendo. 

Ataques causam dano igual ao dano relativo a
FOR, porém podem ser feitos a até 3 metros de distância. 

Gastando 2 PE pode esticar um dos membros a até 3 metros.  Ataques causam dano igual ao dano relativo a
FOR, porém podem ser feitos a até 3 metros de distância. Pode voar a um Deslocamento
de 10 m/s ou 36km/hora. Solta um raio de fogo a até 10 metros que causa 2d6+3 de
dano.
Corpo de Gelo (CON)
5 pontos: O corpo do personagem é feito de gelo. Ganha armadura natural de RD
5. Não sofre penalidades por ferimentos,mas pode ter membros amputados. Toque
causa 1d6 de dano extra de gelo. Gastando 2 PE pode esticar um dos membros a até 3
metros. Se chegar a 0 PVs morre, com o corpo dissolvendo. Ataques causam dano igual
ao dano relativo a FOR, porém podem ser feitos a até 3 metros de distância. Solta um
raio de gelo a até 10 metros que causa 2d6+3 de dano.

\textbf{Corpo de Metal (CON)}:
5 pontos: O corpo do personagem é feito de metal. Ganha armadura natural de
RD 10. Imune a dano por projéteis, se o total de dano for menor que 20. Não sofre
penalidades por ferimentos,mas pode ter membros amputados. Causa 1d6+FOR+4 de
dano em ataques corporais. Se chegar a 0 PVs morre, com o corpo dissolvendo. O peso
normal aumenta em 200 kg.

\textbf{Corpo de Terra (CON)}:
5 pontos: O corpo do personagem é feito de terra. Ganha armadura natural de
RD 10. Não sofre penalidades por ferimentos,mas pode ter membros amputados.
Gastando 2 PE pode esticar um dos membros a até 3 metros. Se chegar a 0 PVs morre,
66com o corpo dissolvendo. Ataques causam dano igual ao dano relativo a FOR, porém
podem ser feitos a até 3 metros de distância.

\textbf{Corpo de Pedra (CON)}:
5 pontos: O corpo do personagem é feito de metal. Ganha armadura natural de
RD 10. Não sofre penalidades por ferimentos,mas pode ter membros amputados. Causa
1d6+FOR+4 de dano em ataques corporais. Se chegar a 0 PVs morre, com o corpo
dissolvendo. O peso normal aumenta em 200 kg.



\textbf{Crescimento (FOR)}: Por alguma razão o personagem consegue controlar seu crescimento.

	\begin{tabu}{|X[c]|X[c]|X[3]|} \tabucline-
		\textbf{Custo} 	& \textbf{Ativação}	&	\textbf{Descrição} \\ \tabucline-
		3	& 	3			& Ganha +3 em FOR,DES e CON (não exceder 10) e aumenta 1,5x seu tamanho original por 10 minutos. \\ \tabucline-
		4	& 	4			& Ganha +4 em FOR,DES e CON (não exceder 10) e aumenta 2x seu tamanho original por 20 minutos. \\ \tabucline-
		5	& 	5			& Ganha +5 em FOR,DES e CON (não exceder 10) e aumenta 3x seu tamanho original por 30 minutos. \\ \tabucline-
		\end{tabu}


\textbf{Defesa Mental (INT)}: Personagem possui treinamento especial de resistência mental.

	\begin{tabu}{|X[c]|X[c]|X[3]|} \tabucline-
		\textbf{Custo} 	& \textbf{Ativação}	&	\textbf{Descrição} \\ \tabucline-
		1	& 	-		& Ganha +4 em testes de INT contra ataques mentais. \\ \tabucline-
		2	& 	-		& Ganha +6 em testes de INT contra ataques mentais. \\ \tabucline-
		3	& 	-		& Ganha +8 em testes de INT contra ataques mentais. \\ \tabucline-
	\end{tabu}


\textbf{Defesa Ampliada (DES)}:Personagem possui treinamento especial em defesa pessoal com equipamentos.

	\begin{tabu}{|X[c]|X[c]|X[3]|} \tabucline-
		\textbf{Custo} 	& \textbf{Ativação}	&	\textbf{Descrição} \\ \tabucline-
		1	& 	-		& Ganha +2 em de DES quando estiver defendendo-se com escudo ou arma de mão. \\ \tabucline-
		2	& 	-		& Ganha +3 em de DES quando estiver defendendo-se com escudo ou arma de mão. \\ \tabucline-
		3	& 	-		& Ganha +4 em de DES quando estiver defendendo-se com escudo ou arma de mão. \\ \tabucline-
		4	& 	-		& Ganha +5 em de DES quando estiver defendendo-se com escudo ou arma de mão. \\ \tabucline-
		5	& 	-		& Ganha +6 em de DES quando estiver defendendo-se com escudo ou arma de mão. \\ \tabucline-
	\end{tabu}


\textbf{Deslocamento Ampliado (DES)}: Devido a um treinamento especial, o personagem consegue deslocar-se de forma mais rápida.

	\begin{tabu}{|X[c]|X[c]|X[3]|} \tabucline-
		\textbf{Custo} 	& \textbf{Ativação}	&	\textbf{Descrição} \\ \tabucline-
		1	& 	-		& Ganha +2 m/s no deslocamento. \\ \tabucline-
		2	& 	-		& Ganha +3 m/s no deslocamento. \\ \tabucline-
		3	& 	-		& Ganha +4 m/s no deslocamento. \\ \tabucline-
		4	& 	-		& Ganha +5 m/s no deslocamento. \\ \tabucline-
		5	& 	-		& Ganha +6 m/s no deslocamento. \\ \tabucline-
	\end{tabu}


\textbf{Direção Absoluta (SAB)}: O personagem possui um senso de direção infalível.

	\begin{tabu}{|X[c]|X[c]|X[3]|} \tabucline-
		\textbf{Custo} 	& \textbf{Ativação}	&	\textbf{Descrição} \\ \tabucline-
		1	& 	-		& Nunca se perde, como se tivesse uma bússola embutida no cérebro, sempre sabe onde é o norte, voltar pelo caminho que fez, etc. \\ \tabucline-
	\end{tabu}



\textbf{Duplicação (INT)}: Personagem domina uma técnica para criar copia(s) idêntica(s) de si mesmo. A(s) cópia(s) ficam ativas por 1 minuto (10 turnos) até desaparecerem, serem dispensadas ou destruídas. Com uma ação, o personagem pode dispensar qualquer uma de suas cópias. Se uma cópia é DESTRUÍDA, o personagem recebe 2 pontos de dano. 

\underline{Concede perícia Duplicação (INT) no mesmo nível da vantagem}: Para controlar a(s) cópia(s) o personagem precisa ser bem sucedido de INT+Duplicação contra uma CD especificada por nível da vantagem, caso contrário a(s) cópia(s) fica(m) fora de controle e o personagem pode repetir o teste no seu turno, na próxima rodada. Personagem receberá 2 pontos de dano por cópia que for destruída antes de ser dispensada.

	\begin{tabu}{|X[c]|X[c]|X[4]|} \tabucline-
		\textbf{Custo} 		& \textbf{Ativação}	&	\textbf{Descrição} \\ \tabucline-
		3	& 	2 PEs		& \begin{itemize}
								\item Cria UMA cópia de si mesmo, que fica ativa por 1 minuto (10 turnos), até ser dispensada ou destruída.
								\item Ganha Perícia Duplicação 3.
								\item Teste de controle com INT+Duplicação vs CD=6.
						  	\end{itemize} \\ \tabucline-

		4	& 	2 PEs		& \begin{itemize}
								\item Cria DUAS cópia de si mesmo, que ficam ativas por 1 minuto (10 turnos), até serem dispensadas ou destruídas.
								\item Ganha Perícia Duplicação 4.
								\item Teste de controle com INT+Duplicação vs CD=8.
							\end{itemize} \\ \tabucline-
		5	& 	2 PEs		& \begin{itemize}
								\item Cria TRÊS cópia de si mesmo, que ficam ativas por 1 minuto (10 turnos), até serem dispensadas ou destruídas.
								\item Ganha Perícia Duplicação 5.
								\item Teste de controle com INT+Duplicação vs CD=10.
							\end{itemize} \\ \tabucline-
	\end{tabu}



\textbf{Duro de Matar (CON)}: O personagem é do tipo casca grossa, difícil de derrubar, tendo pontos de vida extra conforme o nível da vantagem comprada.

	\begin{tabu}{|X[c]|X[c]|X[3]|} \tabucline-
		\textbf{Custo} 	& \textbf{Ativação}	&	\textbf{Descrição} \\ \tabucline-
		1	& 	-		& Ganha +6 PVs extras. \\ \tabucline-
		2	& 	-		& Ganha +8 PVs extras. \\ \tabucline-
		3	& 	-		& Ganha +10 PVs extras. \\ \tabucline-
		4	& 	-		& Ganha +12 PVs extras. \\ \tabucline-
		5	& 	-		& Ganha +14 PVs extras. \\ \tabucline-
	\end{tabu}


\textbf{Energia Extra (CON)}: O personagem veio com ``pilha a mais'', tendo pontos de energia adicional conforme o nível da vantagem comprada.

	\begin{tabu}{|X[c]|X[c]|X[3]|} \tabucline-
		\textbf{Custo} 	& \textbf{Ativação}	&	\textbf{Descrição} \\ \tabucline-
		1	& 	-		& Ganha +2 PEs extras. \\ \tabucline-
		2	& 	-		& Ganha +4 PEs extras. \\ \tabucline-
		3	& 	-		& Ganha +6 PEs extras. \\ \tabucline-
		4	& 	-		& Ganha +8 PEs extras. \\ \tabucline-
		5	& 	-		& Ganha +10 PEs extras. \\ \tabucline-
	\end{tabu}


\textbf{Empatia com Animais (SAB)}: Os animais selvagens acalmam-se com a presença do personagem e não atacarão, a não ser que sejam agredidos ou comandados. O personagem pode tentar controlar o animal com a perícia Controle de Animais.

\underline{Concede perícia Controle de Animais (SAB) no mesmo nível da vantagem}. Personagem conhece técnicas para controlar animais, de forma que o animal controlado irá obedecer aos comandos do personagem.

	\begin{tabu}{|X[c]|X[c]|X[3]|} \tabucline-
		\textbf{Custo} 	& \textbf{Ativação}	&	\textbf{Descrição} \\ \tabucline-
		1	& 	-		& Acalma até 2 animais selvagens.
							\begin{itemize}
								\item Ganha perícia Controle de Animais 1: Pode controlar até 2 animais selvagens.
								\item Teste de controle: SAB+Controle de Animais vs CD=4.
							\end{itemize} \\ \tabucline-
		2	& 	-		& Acalma até 2 animais selvagens.
							\begin{itemize}
								\item Ganha perícia Controle de Animais 2: Pode controlar até 4 animais selvagens.
								\item Teste de controle: SAB+Controle de Animais vs CD=6.
							\end{itemize} \\ \tabucline-
		3	& 	-		& Acalma até 6 animais selvagens.
							\begin{itemize}
								\item Ganha perícia Controle de Animais 3: Pode controlar até 6 animais selvagens.
								\item Teste de controle: SAB+Controle de Animais vs CD=8.
							\end{itemize} \\ \tabucline-
		4	& 	-		& Acalma até 8 animais selvagens.
							\begin{itemize}
								\item Ganha perícia Controle de Animais 4: Pode controlar até 8 animais selvagens.
								\item Teste de controle: SAB+Controle de Animais vs CD=10.
							\end{itemize} \\ \tabucline-
		5	& 	-		& Acalma até 10 animais selvagens.
							\begin{itemize}
								\item Ganha perícia Controle de Animais 5: Pode controlar até 6 animais selvagens.
								\item Teste de controle: SAB+Controle de Animais vs CD=12.
							\end{itemize} \\ \tabucline- 
	\end{tabu}


\textbf{Guardião (não requer atributo)}: Por algum motivo, o personagem possui um Guardião (ser espiritual/criatura mágica/construto nanotecnológico, etc) que pode ser invocado UMA VEZ AO DIA, e que acompanhará o personagem por 4 horas. Observações: 1) O guardião invocado será controlado pelo mesmo jogador do personagem; 2) O personagem pode ter apenas 1 Guardião, ou seja, a vantagem será comprada no nível 5 OU o no nível 4.

** Ao comprar esta vantagem, o jogador \emph{deve} fixar os valores em uma ficha de personagem auxiliar que será usada sempre que o Guardião for invocado.

	\begin{tabu}{|X[c]|X[c]|X[3]|} \tabucline-
		\textbf{Custo} 	& \textbf{Ativação}	&	\textbf{Descrição} \\ \tabucline-
		4	& 	5 PEs		& Invoca o Guardião com: 25 pontos de Atributos, 15 pontos de Perícias e 3 pontos de vantagens, ou     \\ \tabucline-
		5	& 	10 PEs		& Invoca o Guardião com 30 pontos de Atributos, 20 pontos de Perícias, 5 pontos de vantagens (o nível 5 não dá acesso ao nível 4)   \\ \tabucline-	
	\end{tabu}



\textbf{Estômago de Ferro (CON)}: Por algum motivo o personagem não tem problemas estomacais.

	\begin{tabu}{|X[c]|X[c]|X[3]|} \tabucline-
		\textbf{Custo} 	& \textbf{Ativação}	&	\textbf{Descrição} \\ \tabucline-
		2	& 			& Personagem é resistente ao dano causado de intoxicação causada por alimentos. O dano efetivo será sempre a metade do valor obtido nos dados. \\ \tabucline-
		4	& 			& Personagem é imune ao dano causado de intoxicação causada por alimentos. Não há dano ao personagem. \\ \tabucline-
	\end{tabu}


\textbf{Esquiva Ampliada (DES)}: Por algum motivo o personagem é muito ágil e consegue esquivar-se com muita facilidade. \\
\begin{tabu}{|X[c]|X[c]|X[3]|} \tabucline-
	\textbf{Custo} 	& \textbf{Ativação}	&	\textbf{Descrição} \\ \tabucline-
	2	& 	-		& Ganha +3 em testes de DES quando estiver esquivando de ataques. \\ \tabucline-
	3	& 	-		& Ganha +4 em testes de DES quando estiver esquivando de ataques. \\ \tabucline-
	4	& 	-		& Ganha +5 em testes de DES quando estiver esquivando de ataques. \\ \tabucline-
	5	& 	-		& Ganha +6 em testes de DES quando estiver esquivando de ataques. \\ \tabucline-
\end{tabu}

\textbf{Familiares (INT ou CAR)}: Personagem possui uma pequena criatura familiar (animal, espírito, diabrete, etc), inteligente, que o segue para todos os lugares, entende comandos simples e pode levar mensagens por ele. O familiar possui 10 pontos de atributos e 10 pontos de perícia. \\
\begin{tabu}{|X[c]|X[c]|X[3]|} \tabucline-
	\textbf{Custo} 	& \textbf{Ativação}	&	\textbf{Descrição} \\ \tabucline-
	\multirow{2}{*}{2}	& 	2 PEs	& Personagem pode ver pelos olhos do familiar por 10 minutos. \\ \tabucline{2-3}
						& 	2 PEs	& Personagem pode fazer contato telepático com familiar para enviar e receber mensagens curtas (frase com uma oração). \\ \tabucline-
\end{tabu}

\textbf{Favor Divino (CAR)}: Personagem consegue usar o poder de sua fé para receber auxilio divino.
\begin{tabu}{|X[c]|X[c]|X[3]|} \tabucline-
	\textbf{Custo} 	& \textbf{Ativação}	&	\textbf{Descrição} \\ \tabucline-
	3	& 			& A cada duas sessão (ou a cada 4 horas de jogo), o personagem pode receber +6 em uma rolagem de dados ou fazer uma pergunta objetiva, relacionada a cena, ao seu deus. No caso do personagem fazer a pergunta, cabe responder de forma objetiva. \\ \tabucline-
\end{tabu}

\textbf{Flexibilidade (DES)}: Por algum motivo o personagem possui uma flexibilidade muito grande.
\begin{tabu}{|X[c]|X[c]|X[3]|} \tabucline-
	\textbf{Custo} 	& \textbf{Ativação}	&	\textbf{Descrição} \\ \tabucline-
	1	& 			& Ganha +2 em testes para escapar de agarrões, entrar em locais estreitos ou executar tarefas onde a flexibilidade é necessária. \\
	3	& 			& Ganha +4 em testes para escapar de agarrões, entrar em locais estreitos ou executar tarefas onde a flexibilidade é necessária. \\ \tabucline{2-3}
\end{tabu}

\textbf{Engenhoqueiro (INT)}: Personagem tem a habilidade para criar engenhocas com materiais diversos (conforme o universo ficcional). O personagem carrega uma bolsa com ferramentas e materiais que vai encontrando ou obtendo de alguma forma. As engenhocas são como pequenos robôs (se aventura tiver componentes eletrônicos/mecânicos), autômatos a vapor, constructos alquímicos, etc.

** Observação: É importante que as perícias e vantagens da engenhoca sejam compatíveis com o universo ficcional. \\
\begin{tabu}{|X[c]|X[c]|X[3]|} \tabucline-
	\textbf{Custo} 	& \textbf{Ativação}	&	\textbf{Descrição} \\ \tabucline-
	1	& 	-		& Em 15 minutos, cria engenhoca com 10 pontos de atributo, 5 de perícia e 1 ponto de vantagens. Atributo base para dano FOR. \\ \tabucline-
	2	& 	-		& Em 15 minutos, cria engenhoca com 15 pontos de atributo, 10 de perícia e 2 ponto de vantagens. Atributo base para dano FOR. \\ \tabucline-
	3	& 	-		& Em 30 minutos, cria engenhoca com 20 pontos de atributo, 15 de perícia e 3 ponto de vantagens. Atributo base para dano FOR ou DES. \\ \tabucline-
	4	& 	-		& Em 30 minutos, cria engenhoca com 25 pontos de atributo, 20 de perícia e 4 ponto de vantagens. Atributo base para dano FOR ou DES. \\ \tabucline-
	5	& 	-		& Em 60 minutos, cria engenhoca com 30 pontos de atributo, 25 de perícia e 5 ponto de vantagens. Atributo base para dano FOR, DES ou PODER. \\ \tabucline-
\end{tabu}





\textbf{Raio Elétrico (PODER)}: Por algum motivo, o personagem tem a capacidade de produzir raios direcionados.


Personagem tem a habilidade para criar engenhocas com materiais diversos (conforme o universo ficcional). O personagem carrega uma bolsa com ferramentas e materiais que vai encontrando ou obtendo de alguma forma. As engenhocas são como pequenos robôs (se aventura tiver componentes eletrônicos/mecânicos), autômatos a vapor, constructos alquímicos, etc.

** Observação: É importante que as perícias e vantagens da engenhoca sejam compatíveis com o universo ficcional. \\
\begin{tabu}{|X[c]|X[c]|X[3]|} \tabucline-
	\textbf{Custo} 	& \textbf{Ativação}	&	\textbf{Descrição} \\ \tabucline-
	1	& 	-		& Em 15 minutos, cria engenhoca com 10 pontos de atributo, 5 de perícia e 1 ponto de vantagens. Atributo base para dano FOR. \\ \tabucline-
	2	& 	-		& Em 15 minutos, cria engenhoca com 15 pontos de atributo, 10 de perícia e 2 ponto de vantagens. Atributo base para dano FOR. \\ \tabucline-
	3	& 	-		& Em 30 minutos, cria engenhoca com 20 pontos de atributo, 15 de perícia e 3 ponto de vantagens. Atributo base para dano FOR ou DES. \\ \tabucline-
	4	& 	-		& Em 30 minutos, cria engenhoca com 25 pontos de atributo, 20 de perícia e 4 ponto de vantagens. Atributo base para dano FOR ou DES. \\ \tabucline-
	5	& 	-		& Em 60 minutos, cria engenhoca com 30 pontos de atributo, 25 de perícia e 5 ponto de vantagens. Atributo base para dano FOR, DES ou PODER. \\ \tabucline-
\end{tabu}
1 ponto: Com um gasto de 1PE o personagem solta um Raio Elétrico que dá até
1d6 de dano (o personagem controla o dano), com alcance de 5 metros, alvo faz teste
normal de CON para não ficar desmaiar por 1 rodada. Perícia Raio Elétrico 1
2 pontos: Com um gasto de 2PEs o personagem solta um Raio Elétrico que dá até
2d6 de dano (o personagem controla o dano), com alcance de 10 metros, alvo faz teste
normal de CON para não ficar desmaiar por 1 rodada. Perícia Raio Elétrico 2
3 pontos: Com um gasto de 3PEs o personagem solta um Raio Elétrico que dá até
3d6 de dano (o personagem controla o dano), com alcance de 30 metros, alvo faz teste
normal de CON para não ficar desmaiar por 1 rodada. Perícia Raio Elétrico 3
4 pontos: Com um gasto de 4PEs o personagem solta um Raio Elétrico que dá até
4d6 de dano (o personagem controla o dano), com alcance de 60 metros, alvo faz teste
normal de CON para não ficar desmaiar por 1 rodada. Perícia Raio Elétrico 4
5 pontos: Com um gasto de 5PEs o personagem solta um Raio Elétrico que dá até
5d6 de dano (o personagem controla o dano), com alcance de 120 metros, alvo faz teste
normal de CON para não ficar desmaiar por 1 rodada. Perícia Raio Elétrico 5

\end{small}



