\chapter{\label{ch:classes}Classes de Personagem}

Classes são como funções que o personagem exerce/assume durante uma aventura. Assim, ao criar seu personagem, o jogador define qual classe irá interpretar durante a aventura. Destacamos que o Sistema +2d6@IM não obriga o uso das classes, portante os jogadores podem criar seus personagens sem classe.

As seções a seguir apresentam as principais classes do sistema +2d6@IM, mas outras podem ser criadas livremente.


\section{\label{sec:aventMedievais}Classes para aventuras fantásticas}

\subsection*{Arqueiros}
O arqueiro é uma classe de personagem que domina a arte do combate à distância. Com um arco e flecha em mãos, agilidade e conhecimento tático do terreno, fazem do arqueiro um personagem formidável, capaz de lançar com vantagem a partir de locais estratégicos e privilegiados. 

\begin{table}[htb]
	\centering\smaller
	\emph{Inteligências importantes para Arqueiros.}
	\begin{tabu} to \textwidth {|X[c 0.5]|X[1]|X[3]|} \tabucline-
		\textbf{Ordem}	& \textbf{Inteligência}	&	\textbf{Descrição}	\\ \tabucline-
		1º		& \emph{Corporal-cinestésica - Destreza}  	& Confere a capacidade para o manejo de armas a distância.	\\ \tabucline-
		2º		& \emph{Espacial} & Confere a capacidade de reconhecer e escolher bem o terreno onde estiver, seja em combate ou não.\\ \tabucline-
	\end{tabu}
\end{table}

\begin{table}[htb]
	\centering\smaller
	\emph{Equipamentos importantes para Arqueiros.}
	\begin{tabu} to \textwidth {|X[0.5]|X[3]|} \tabucline-
		\textbf{Equipamentos}	&	\textbf{Descrição}	\\ \tabucline-
		\emph{Armaduras}  	& Apenas armaduras leves de material não reflexivo. Metal ou outro material que reflete a luz podem dificultar ações de camuflagem ou furtivas.	\\ \tabucline-
		\emph{Armas} & Preferem arco longo, mas sabem/podem utilizar qualquer tipo de arma a distância como arcos, besta ou dardos. As flechas serão carregadas em uma aljava, com capacidade para 20 a 25 flechas ou virotes.\\ \tabucline-
	\end{tabu}
\end{table}


\subsection*{Bárbaros}
O bárbaro é um guerreiro primitivo e indomável, que canaliza sua fúria nas batalhas. Ligado profundamente aos instintos mais básicos, ele enfrenta seus inimigos com uma ferocidade inigualável. Seu corpo é uma arma que possui grande resistência a golpes e é capaz de infligir danos devastadores. Os bárbaros são conhecidos por sua resistência física e agilidade.

\begin{table}[htb]
	\centering\smaller
	\emph{Inteligências importantes para Bárbaros.} \\
	\begin{tabu} to \textwidth {|X[c 0.5]|X[1]|X[3]|} \tabucline-
		\textbf{Ordem}	& \textbf{Inteligência}	&	\textbf{Descrição}	\\ \tabucline-
		1º		& \emph{Corporal-cinestésica - Força}  	& Confere a capacidade de luta no sentido de causar dano.	\\ \tabucline-
		2º		& \emph{Corporal-cinestésica - Destreza} & Confere a capacidade de esquiva e mobilidade durante combates. \\ \tabucline-
		3º		& \emph{Vitalidade} & Devido a natureza selvagem do bárbaro e sua aptidão ao combate corpo-a-corpo, recomenda-se destinar pontos para Vitalidade, pois a quantidade de vida é proporcional à este atributo.\\ \tabucline-
	\end{tabu}
\end{table}

\begin{table}[htb]
	\centering\smaller
	\emph{Equipamentos importantes para Bárbaros.}
	\begin{tabu} to \textwidth {|X[0.5]|X[3]|} \tabucline-
		\textbf{Equipamentos}	&	\textbf{Descrição}	\\ \tabucline-
		\emph{Armaduras}  	& Apenas armaduras leves para facilitar sua mobilidade.	\\ \tabucline-
		\emph{Armas} & Preferem machados de guerra, mas são capazes de utilizar qualquer tipo de arma.\\ \tabucline-
	\end{tabu}
\end{table}

\subsection*{Clérigos}
O clérigo dedica sua existência ao sagrado e cultiva uma fé inabalável em uma divindade superior existente na história. Este deus concede força moral e poder mágico ao clérigo, que luta incansavelmente contra as hordas do mal e criaturas conspurcadas pelas sombras malignas. Entretanto, clérigos nunca causarão ferimentos graves a outras criaturas, a não ser que tenha certeza de que ela esteja a trabalho das forças do mal.

Além de seus poderes curativos, armas sagradas e magias divinas, o clérigo possui grande sabedoria e conhecimento das escrituras sagradas (do seu deus), sendo, algumas vezes, um tipo de guia espiritual para seu grupo, oferecendo conselhos e apoio em momentos de necessidade.

\begin{table}[htb]
	\centering\smaller
	\emph{Inteligências importantes para Clérigos.} \\
	\begin{tabu} to \textwidth {|X[c 0.5]|X[1]|X[3]|} \tabucline-
		\textbf{Ordem}	& \textbf{Inteligência}	&	\textbf{Descrição}	\\ \tabucline-
		1º		& \emph{Intrapessoal/Existencial}  	& Confere a capacidade de de comunhão espiritual com sua divindade, por onde recebe sua força e magia. 	\\ \tabucline-
		2º		& \emph{Interpessoal} & Confere a capacidade de estabelecer uma forte relação e grande influência em seus seguidores, que o caracteriza como ``guia espiritual''. \\ \tabucline-
		3º		& \emph{Vitalidade} & Devido a natureza combatente do clérigo e seu forte desejo de destruir todo o mal, recomenda-se destinar pontos para Vitalidade, pois a quantidade de vida é proporcional à este atributo.\\ \tabucline-
	\end{tabu}
\end{table}

\begin{table}[htb]
	\centering\smaller
	\emph{Equipamentos importantes para Bárbaros.}
	\begin{tabu} to \textwidth {|X[0.5]|X[3]|} \tabucline-
		\textbf{Equipamentos}	&	\textbf{Descrição}	\\ \tabucline-
		\emph{Armaduras}  	& Qualquer tipo de armadura, mas preferem armaduras pesadas e reluzentes.	\\ \tabucline-
		\emph{Armas} & Apenas as com dano de \emph{concussão}. Devido aos juramentos feitos à divindade e adoção de doutrinas sagradas, clérigos são impedidos de utilizar qualquer arma afiada (que cause cortes ou perfurações), pois nem mesmo a mais infame criatura merece ter seu sangue derramado.\\ \tabucline-
		\emph{Símbolo sagrado} & Clérigos carregam sempre consigo um artefato ou símbolo sagrado, que foi abençoado por seu deus e que funciona como um foco do seu poder mágico. Por exemplo: medalhão, adaga sagrada, pérola negra, etc... \\ \tabucline-
	\end{tabu}
\end{table}

\subsection*{Druida}
Druidas possuem uma profunda conexão com a Natureza, de onde buscam forças, alimentos, medicamentos e até poderes mágicos elementais. Com um profundo respeito e conhecimento pelo mundo natural e suas criaturas, eles possuem a capacidade de convocar os poderes da terra, do ar, do fogo e da água para os auxiliar em suas jornadas. 

Seus poderes mágicos permitem que ele se transforme em animais, comunique-se com plantas e controlem os 4 elementos. Além de suas habilidades mágicas, o druida é um ferrenho defensor da natureza, utilizando suas habilidades para proteger florestas, rios e criaturas selvagens. Muitas vezes, os druidas são vistos como sábios e conselheiros, oferecendo a seus conhecimentos sobre as plantas medicinais e os ciclos da natureza para ajudar aqueles que cruzam seu caminho.

Devido a sua profunda conexão com a Natureza, os druidas sabem que a terra foi ferida, tendo suas entranhas rasgadas para que o metal saísse de suas profundezas, portando, não se permitem usar/vestir nada feito de metal.

\begin{table}[htb]
	\centering\smaller
	\emph{Inteligências importantes para Druidas.} \\
	\begin{tabu} to \textwidth {|X[c 0.5]|X[1]|X[3]|} \tabucline-
		\textbf{Ordem}	& \textbf{Inteligência}	&	\textbf{Descrição}	\\ \tabucline-
		1º		& \emph{Naturalista}  	& Confere conhecimentos da Natureza, suas criaturas, poderes elementais, etc. 	\\ \tabucline-
		2º		& \emph{Espacial} & Confere conhecimento dos lugares e espaços onde há Natureza selvagem. \\ \tabucline-
	\end{tabu}
\end{table}

\begin{table}[htb]
	\centering\smaller
	\emph{Equipamentos importantes para Druidas.}
	\begin{tabu} to \textwidth {|X[0.5]|X[3]|} \tabucline-
		\textbf{Equipamentos}	&	\textbf{Descrição}	\\ \tabucline-
		\emph{Armaduras} & Apenas armaduras e escudos feitos de elementos naturais, como madeira, cipós, etc.	\\ \tabucline-
		\emph{Armas} & Apenas simples como clavas e bastões ou armas fabricadas com galhos, pedras e ossos. \\ \tabucline-
		\emph{Cajado} & Druidas portam sempre um cajado feito nobres elementos naturais, como um majestoso galho cedido pela árvore anciã e ricamente adornado com penugens retiradas de ninhos abandonados. O cajado é um catalisador do poder elemental presente na Natureza, permitindo ao Druida conjurar suas magias. \\ \tabucline-
	\end{tabu}
\end{table}

\subsection*{Bardo}
O bardo é um artista itinerante que utiliza a magia proveniente da arte, da música ou da poesia, para inspirar e influenciar aqueles ao seu redor. Com um instrumento em mãos, ele é capaz de compor canções épicas, contar histórias fascinantes e até mesmo manipular as emoções dos outros. 

A magia do bardo é única, pois ela está intimamente ligada à sua criatividade e à força de suas palavras. Seja acalmando uma criatura selvagem com uma melodia suave ou fortalecendo seus aliados na batalha com uma canção inspiradora, o bardo é um membro valioso em qualquer grupo de aventureiros. Bardos também podem ser ótimos encrenqueiros e grandes sedutores. 

\begin{table}[htb]
	\centering\smaller
	\emph{Inteligências importantes para Bardos.} \\
	\begin{tabu} to \textwidth {|X[c 0.5]|X[1]|X[3]|} \tabucline-
		\textbf{Ordem}	& \textbf{Inteligência}	&	\textbf{Descrição}	\\ \tabucline-
		1º		& \emph{Artística/Musical}  	& Confere habilidades de interpretação, representação, canto, domínio de instrumentos musicais, etc. 	\\ \tabucline-
		2º		& \emph{Interpessoal} & Confere habilidades de interação social, carisma e sedução. \\ \tabucline-
	\end{tabu}
\end{table}

\begin{table}[htb]
	\centering\smaller
	\emph{Equipamentos importantes para Bardos.}
	\begin{tabu} to \textwidth {|X[0.5]|X[3]|} \tabucline-
		\textbf{Equipamentos}	&	\textbf{Descrição}	\\ \tabucline-
		\emph{Armaduras} & Apenas armaduras leves.	\\ \tabucline-
		\emph{Armas} & Armas simples como adagas, rapieiras, espadas curtas ou bestas de mão. \\ \tabucline-
		\emph{Instrumento musical} & Um instrumento como flauta, alaúde, violino, sanfona ou outro qualquer, desde que seja portátil em uma aventura. O instrumento é usado para diversos fins, principalmente para inspirar companheiros de jornada. \\ \tabucline-
	\end{tabu}
\end{table}

\subsection*{Guerreiro}
Guerreiros são pessoas que receberam intenso treinamento militar. Sabem utilizar qualquer equipamentos de combate e também podem ser habilidosos estrategistas ou comandantes. 

Sua coragem inabalável e sua habilidade de liderar inspirem seus aliados, ao mesmo tempo que sua presença intimidante aterroriza os oponentes. Guerreiros são a linha de frente no combate, onde lutam para proteger aqueles que não podem se defender por si mesmos. 

\begin{table}[htb]
	\centering\smaller
	\emph{Inteligências importantes para Guerreiros.} \\
	\begin{tabu} to \textwidth {|X[c 0.5]|X[1]|X[3]|} \tabucline-
		\textbf{Ordem}	& \textbf{Inteligência}	&	\textbf{Descrição}	\\ \tabucline-
		1º		& \emph{Corporal-cinestésica - Força}  	& Confere habilidades de interpretação, representação, canto, domínio de instrumentos musicais, etc. 	\\ \tabucline-
		2º		& \emph{Interpessoal} & Confere habilidades de liderança, comando, intimidação, etc. \\ \tabucline-
		3º 		& \emph{Vitalidade} & Devido a natureza combatente do guerreiro e seus esforços em proteger os mais fracos, recomenda-se destinar pontos para Vitalidade, pois a quantidade de vida é proporcional à este atributo. \\ \tabucline-
	\end{tabu}
\end{table}

\begin{table}[htb]
	\centering\smaller
	\emph{Equipamentos importantes para Guerreiros.}
	\begin{tabu} to \textwidth {|X[0.5]|X[3]|} \tabucline-
		\textbf{Equipamentos}	&	\textbf{Descrição}	\\ \tabucline-
		\emph{Armaduras} & Qualquer armadura.	\\ \tabucline-
		\emph{Armas} & Qualquer arma. \\ \tabucline-
	\end{tabu}
\end{table}

\subsection*{Ladino}
O ladino é um mestre nas artes da infiltração e dissimulação, com uma habilidade ímpar para se mover nas sombras e enganar seus inimigos. É capaz de abrir fechaduras, desarmar armadilhas e se mover silenciosamente por locais perigosos. Além de suas habilidades físicas, muitos ladinos possuem um conhecimento profundo de diversas culturas e línguas, permitindo-lhes se infiltrar em praticamente qualquer sociedade. 

Com sua natureza astuta e calculista, costuma evitar o combate direto, mas ainda assim é um aliado valioso para qualquer grupo de aventureiros, capaz de obter informações cruciais, desviar perigos e realizar missões perigosas.

\begin{table}[htb]
	\centering\smaller
	\emph{Inteligências importantes para Ladinos.} \\
	\begin{tabu} to \textwidth {|X[c 0.5]|X[1]|X[3]|} \tabucline-
		\textbf{Ordem}	& \textbf{Inteligência}	&	\textbf{Descrição}	\\ \tabucline-
		1º		& \emph{Corporal-cinestésica - Destreza}  	& Confere habilidades para o manuseio de ferramentas delicadas e precisas, bem como de movimentar-se de forma sorrateira e imperceptível. 	\\ \tabucline-
		2º		& \emph{Linguística/Verbal} & Confere conhecimentos de diversas línguas e culturas, incluindo dialetos, gírias e jargões do submundo. \\ \tabucline-
	\end{tabu}
\end{table}

\begin{table}[htb]
	\centering\smaller
	\emph{Equipamentos importantes para Ladinos.}
	\begin{tabu} to \textwidth {|X[0.5]|X[3]|} \tabucline-
		\textbf{Equipamentos}	&	\textbf{Descrição}	\\ \tabucline-
		\emph{Armaduras} & Apenas armaduras leves.	\\ \tabucline-
		\emph{Armas} & Armas simples como adagas, rapieiras, espadas curtas ou bestas de mão. \\ \tabucline-
		\emph{Kit de trabalho} & Ladinos costuma andar sempre com sua pequena maleta de utensílios variados, utilizados para arrombar fechaduras, desarmar armadilhas, descobrir códigos de cofres, etc. \\ \tabucline-
	\end{tabu}
\end{table}

\subsection*{Mago}
O mago é um estudioso das forças místicas do universo, sendo capaz de manipular a trama mágica que nos rodeia. São capazes de criar ilusões, manipular e transformar elementos. Devido aos seus longos anos de estudo, possuem profundo conhecimento de linguagens antigas, runas, rituais, etc. 

O mago possui um grimório, onde todos seus feitiços estão meticulosamente registrados, por esse motivo, irá proteger\footnote{Um mago cuidadoso sempre possui uma cópia de seu grimório guardada em local seguro e conhecido apenas por ele.} seu grimório aconteça o que acontecer.

Para conjurar suas magias o mago precisa ter mãos livres (para fazer os sinais que a magia exige), precisa poder falar (para pronunciar as palavras mágicas) e precisa do seu foco arcano, onde concentra a energia da trama mágica que nos rodeia.

\begin{table}[htb]
	\centering\smaller
	\emph{Inteligências importantes para Magos.} \\
	\begin{tabu} to \textwidth {|X[c 0.5]|X[1]|X[3]|} \tabucline-
		\textbf{Ordem}	& \textbf{Inteligência}	&	\textbf{Descrição}	\\ \tabucline-
		1º		& \emph{Linguística/Verbal}  	& Confere habilidades de compreender e utilizar diversos idiomas, inclusive mágicos. 	\\ \tabucline-
		2º		& \emph{Lógico/Matemática} & Confere habilidades de compreender e manipular as complexas equações arcanas.\\ \tabucline-
	\end{tabu}
\end{table}

\begin{table}[htb]
	\centering\smaller
	\emph{Equipamentos importantes para Magos.}
	\begin{tabu} to \textwidth {|X[0.5]|X[3]|} \tabucline-
		\textbf{Equipamentos}	&	\textbf{Descrição}	\\ \tabucline-
		\emph{Grimório} & Livro com o registro de todas as magias que o mago conhece.	\\ \tabucline-
		\emph{Foco arcano} & É um artefato que concentra seu poder mágico, podendo ser um Cajado, Varinha, Orbe, Amuletos ou Talismãs. \\ \tabucline-
		\emph{Armadura} & Não usam armaduras porque atrapalham seus movimentos para conjurar magias. \\ \tabucline-
		\emph{Armas} & Apenas armas simples. \\ \tabucline-
	\end{tabu}
\end{table}


\subsection*{Paladino}
O paladino é um guerreiro sagrado, imbuído de um forte senso de justiça e honra. Ele é um defensor incansável dos fracos e oprimidos, combatendo as forças do mal com uma espada na mão e uma fé inabalável no coração. 

A força do paladino reside não apenas em sua proeza física, mas também em sua conexão com uma força superior, que lhe concede poderes divinos e a capacidade de curar feridas e abençoar seus aliados. Os paladinos são conhecidos por sua coragem inabalável, sua lealdade e seu compromisso com um código de honra rígido. 

Com sua armadura brilhante e sua aura de santidade, o paladino inspira confiança e esperança em todos aqueles que cruzam seu caminho.

\begin{table}[htb]
	\centering\smaller
	\emph{Inteligências importantes para Paladinos.} \\
	\begin{tabu} to \textwidth {|X[c 0.5]|X[1]|X[3]|} \tabucline-
		\textbf{Ordem}	& \textbf{Inteligência}	&	\textbf{Descrição}	\\ \tabucline-
		1º		& \emph{Corporal-cinestésica (Força)} & Confere a força para manusear sua espada contra as forças do mal. \\ \tabucline-
		2º		& \emph{Intrapessoal/Existencial}  	& Confere a conexão divina para suas magias de cura e proteção. 	\\ \tabucline-
	\end{tabu}
\end{table}

\begin{table}[htb]
	\centering\smaller
	\emph{Equipamentos importantes para Paladinos.}
	\begin{tabu} to \textwidth {|X[0.5]|X[3]|} \tabucline-
		\textbf{Equipamentos}	&	\textbf{Descrição}	\\ \tabucline-
		\emph{Armaduras} & Qualquer tipo de armadura (normalmente reluzente e com símbolos sagrados).	\\ \tabucline-
		\emph{Armas} & Qualquer tipo de arma, preferindo as espadas longas ou bastarda. \\ \tabucline-
	\end{tabu}
\end{table}


\subsection*{Rastreador}
Rastreadores são exploradores experientes, conhecem profundamente a natureza e o comportamento dos animais. Com uma percepção aguçada e habilidades de sobrevivência excepcionais, eles navegam por terras selvagens e perigosas com facilidade.

São capazes de rastrear suas presas por longas distâncias e emboscar seus inimigos com muita precisão. Possuem também um profundo conhecimento das plantas, dos animais e dos espíritos da natureza, aproveitando os recursos naturais para sobreviver. 

\begin{table}[htb]
	\centering\smaller
	\emph{Inteligências importantes para Rastreadores.} \\
	\begin{tabu} to \textwidth {|X[c 0.5]|X[1]|X[3]|} \tabucline-
		\textbf{Ordem}	& \textbf{Inteligência}	&	\textbf{Descrição}	\\ \tabucline-
		1º		& \emph{Espacial}  	& Confere habilidades de compreensão e reconhecimento do espaço em que está, bem como de localização. 	\\ \tabucline-
		2º		& \emph{Naturalista} & Confere o conhecimento da natureza. Consegue identificar pegadas de animais, reconhecer plantas, encontrar fontes de água pura, etc. \\ \tabucline-
	\end{tabu}
\end{table}

\begin{table}[htb]
	\centering\smaller
	\emph{Equipamentos importantes para Rastreadores.}
	\begin{tabu} to \textwidth {|X[0.5]|X[3]|} \tabucline-
		\textbf{Equipamentos}	&	\textbf{Descrição}	\\ \tabucline-
		\emph{Armaduras} & Somente as leves e naturais.	\\ \tabucline-
		\emph{Armas} & Qualquer arma comum. \\ \tabucline-
	\end{tabu}
\end{table}


%
%\subsection*{generico}
%texto aqui
%
%\begin{table}[htb]
%	\centering\smaller
%	\emph{Inteligências importantes para ???.} \\
%	\begin{tabu} to \textwidth {|X[c 0.5]|X[1]|X[3]|} \tabucline-
%		\textbf{Ordem}	& \textbf{Inteligência}	&	\textbf{Descrição}	\\ \tabucline-
%		1º		& \emph{???}  	& Confere habilidades ???. 	\\ \tabucline-
%		2º		& \emph{???} & Confere ???? \\ \tabucline-
%	\end{tabu}
%\end{table}
%
%\begin{table}[htb]
%	\centering\smaller
%	\emph{Equipamentos importantes para ???.}
%	\begin{tabu} to \textwidth {|X[0.5]|X[3]|} \tabucline-
%		\textbf{Equipamentos}	&	\textbf{Descrição}	\\ \tabucline-
%		\emph{Armaduras} & ????.	\\ \tabucline-
%		\emph{Armas} & ????. \\ \tabucline-
%		\emph{?????} & ????. \\ \tabucline-
%	\end{tabu}
%\end{table}
%

%
%\subsection*{\textcolor{red}{Alquimistas}}
%Os alquimistas são estudiosos que unem ciência e natureza, buscando a transmutação matéria e energia. Na busca por dominar a alquimia, estão constantemente procurando saber mais sobre os elementos e da natureza.
%
%São capazes de criar poções e elixires, que podem aumentar a força, curar ferimentos 
%
%Utilizando seus instrumentos 
%
%Inteligências recomendadas:
%\begin{itemize}
%	\item \textbf{Principal}: Lógico-Matemática
%	\item \textbf{Secundária}: Naturalista
%\end{itemize}
%
%
%
%\subsection*{\textcolor{red}{Artífice}}
%Inteligências 
%\begin{itemize}
%	\item \textbf{Principal}: Artística/Musical
%	\item \textbf{Secundária}: Lógico-matemática
%\end{itemize}
%
%\subsection*{Monge}
%Inteligências 
%\begin{itemize}
%	\item \textbf{Principal}: Corporal-cinestésica - Destreza
%	\item \textbf{Secundária}: Intrapessoal/Existencial
%\end{itemize}
%
%\subsection*{Psiônicos}
%Inteligências 
%\begin{itemize}
%	\item \textbf{Principal}: Interpessoal
%	\item \textbf{Secundária}: Linguística/Verbal
%\end{itemize}
%

%
%\section{\label{sec:aventModernas}Aventuras Modernas}