\chapter{\label{ch:rpg}RPG de Mesa}

O RPG de Mesa\footnote{Em inglês Role Playing Game} é um jogo cooperativo e de interpretação de papéis, idealizado por Gary Gygax (1974). Em um RPG há uma história com vários personagens, mistérios e conflitos. Os jogadores participam da história, interagindo com os personagens e com o ambiente/cenário, fazendo a história evoluir.

Um dos jogadores (chamado Mestre) narra a história, controla (e interpreta) os personagens do Mestre (PdM) e descreve/narra os eventos que acontecem conforme os outros personagens agem. Os demais jogadores controlam (e interpretam) os personagens do jogador (PdJ), que interagem com a história narrada e com os eventos decorridos. 

Dentro do jogo os personagens podem fazer praticamente qualquer coisa, como conversar com outros personagens, manipular objetos, resolver ou criar conflitos, viajar para lugares distantes, inventar máquinas, experimentar poções, ou qualquer outra coisa que seja possível dentro do universo da história e seja compatível com as capacidades do personagem.

O RPG de Mesa é uma forma de entretenimento que vai além do simples jogo, sendo também uma poderosa ferramenta para o desenvolvimento da criatividade, pois envolve atividades como criação de personagens, de narrativas, de cenários fictícios, etc, estimulando a imaginação de maneira profunda e envolvente.

\section{\label{sec:necessario}O que é preciso para jogar RPG}
De forma geral é preciso:
\begin{itemize}
	\item \textbf{Grupo de Jogadores}: Reúna um grupo de amigos ou pessoas interessadas.
	\item \textbf{Um Mestre}: É o jogador responsável por narrar (e às vezes criar) a história, controlar os PdM's, definir a dificuldade dos desafios e manter o jogo em andamento. O Mestre também toma decisões sobre o jogo, narrando as consequências das ações dos jogadores.
	\item \textbf{Um Sistema de RPG}: Um conjunto de regras que define como criar personagens, resolver conflitos e executar a mecânica do jogo. Este Livro de Regras apresenta o \underline{Sistema +2d6@ifsul, que é simples, livre e gratuito}. 	
	No mercado, existem vários outros sistemas, como: \emph{Tormenta 20}\footnote{\url{https://jamboeditora.com.br/categoria/rpg/tormenta20-rpg/}}, \emph{Ordem Paranormal}\footnote{\url{https://jamboeditora.com.br/produto/ordem-paranormal-rpg/}} e \emph{Old Dragon} 2\footnote{\url{https://www.burobrasil.com/produtos/old-dragon/}}. 
	\item \textbf{Fichas de Personagem}: Cada jogador precisará de uma ficha na qual são registrados atributos, habilidades, história e outras informações sobre seu personagem. O livro de regras geralmente fornece modelos de fichas.
	\item \textbf{Dados de RPG}: Os dados são usados para alguns tipos de testes durante o jogo. O Sistema +2d6@ifsul requer apenas 2 dados de 6 faces, sendo uma alternativa de fácil acesso.
	\item \textbf{Imaginação e Criatividade}: RPGs são jogos baseados em interações criativas, imaginadas (e vivenciadas) no Teatro da Mente.
	\item \textbf{Tempo e Respeito}: Jogos de RPG exigem respeito mútuo e alguma dedicação de tempo.
\end{itemize}

