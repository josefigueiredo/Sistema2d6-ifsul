\chapter{\label{ch:rpg}RPG de Mesa}

O RPG de Mesa\footnote{Em inglês Role-playing Game} é um jogo cooperativo e de interpretação de papéis, idealizado por Gary Gygax (1974), que criou o sistema \emph{Dungeons \& Dragons}. Depois de Gygax surgiram diversos outros sistemas de RPG que podem ser pesquisados facilmente em uma busca a Internet. 

A seguir apresentamos uma pequena lista com alguns sistemas de RPG brasileiros:
\begin{itemize}
	\item \emph{Tormenta 20} - disponível em: \\ \url{https://jamboeditora.com.br/categoria/rpg/tormenta20-rpg/};
	\item \emph{Ordem Paranormal} - disponível em: \\
	\url{https://jamboeditora.com.br/produto/ordem-paranormal-rpg/}
	\item \emph{Old Dragon 2} - disponível em: \\ \url{https://www.burobrasil.com/produtos/old-dragon/}.
	\item \emph{Defensores de Tóquio (3D\&T)} - disponível em: \\ \url{https://jamboeditora.com.br/categoria/rpg/3det/};
	\item \emph{Mojubá RPG} - disponível em: \\ \url{https://ludopedia.com.br/jogo/mojuba-rpg};
\end{itemize}

Em um jogo de RPG há uma história, com vários personagens, mistérios, conflitos, magia, mundos distantes ou fantásticos, etc. Os jogadores participam do jogo, controlando e interpretando seus personagens, enquanto interagem com outros personagens da narrativa, com o ambiente/cenário, fazendo a história evoluir. 

Um dos jogadores (chamado Mestre) narra a história, controla (e interpreta) os personagens do Mestre (PdM) e descreve/narra os eventos que acontecem conforme os outros personagens agem. 

Dentro do jogo os personagens podem fazer praticamente qualquer coisa, como conversar com outros personagens, manipular objetos, resolver ou criar conflitos, viajar para lugares distantes, inventar máquinas de poções, ou qualquer outra coisa que seja possível dentro do universo ficcional da história e seja compatível com as capacidades do personagem.

Utilizando a imaginação e criatividade, os jogadores conduzem seus personagens dentro da história enquanto interagem com os outros personagens, com o ambiente, com mecanismos e armadilhas, tesouros e tudo o mais que possa existir em uma boa história. Por estarem inseridos no universo ficcional, os personagens também estão sujeitos aos grandes eventos desse mundo como cataclismos, invasões alienígenas, profecias, maldições , lendas, etc. Além disso, a ação destes mesmos personagens, pode também causar mudanças no mundo ficcional da história. 

O RPG de mesa é um jogo colaborativo onde não existe um vencedor, ou seja, os participantes não competem entre si. Na aventura, todos precisam colaborar para obterem sucesso na resolução do(s) conflito(s) da narrativa. Em teoria não há limites para a exploração destes aspectos e, em função disso, é lógico inferir que o RPG proporciona um forte estímulo para o desenvolvimento de habilidades como imaginação, criatividade, trabalho em equipe, resolução de problemas, entre outros.

Além disso, ao adentrar no universo do RPG de Mesa, naturalmente os jogadores desenvolvem (ou aumentam) o hábito da leitura, mesmo que de forma não convencional. Essa imersão impele o jogador na busca pela compreensão dos sistemas de regras, das histórias pregressas dos personagem ou cenários, e, principalmente, na busca de elementos para enriquecer a própria criatividade.

O RPG de Mesa é uma forma de entretenimento que vai além do simples jogo, sendo também uma poderosa ferramenta para o desenvolvimento da criatividade, pois envolve atividades como criação de personagens, de narrativas, de cenários fictícios, etc, estimulando a imaginação de maneira profunda e envolvente.


\section{\label{sec:necessario}O que é preciso para jogar RPG}
De forma geral é preciso:
\begin{itemize}
	\item \textbf{Grupo de Jogadores}: Reúna um grupo de amigos ou pessoas interessadas.
	\item \textbf{Um Mestre}: É o jogador responsável por narrar (e às vezes criar) a história, controlar os PdM's, definir a dificuldade dos desafios e manter o jogo em andamento. O Mestre também toma decisões sobre o jogo, narrando as consequências das ações dos jogadores.
	\item \textbf{Um Sistema de RPG}: Um conjunto de regras que define como criar personagens, resolver conflitos e executar a mecânica do jogo. Este Livro de Regras apresenta o \underline{Sistema +2d6@IM, que é simples, livre e gratuito}, mas outros sistemas também podem ser utilizados. 
	\item \textbf{Fichas de Personagem}: Cada jogador precisará de uma ficha na qual são registrados atributos, habilidades, história e outras informações sobre seu personagem. O livro de regras geralmente fornece modelos de fichas.
	\item \textbf{Dados de RPG}: Os dados são usados para alguns tipos de testes durante o jogo. O Sistema +2d6@IM requer apenas 2 dados de 6 faces, sendo uma alternativa de fácil acesso.
	\item \textbf{Imaginação e Criatividade}: RPGs são jogos baseados em interações criativas, imaginadas (e vivenciadas) no Teatro da Mente.
	\item \textbf{Tempo e Respeito}: Jogos de RPG exigem respeito mútuo e alguma dedicação de tempo.
\end{itemize}

