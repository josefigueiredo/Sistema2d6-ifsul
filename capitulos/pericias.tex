\chapter{\label{ch:pericias}Perícias}

Perícias representam as especialidades e poderes especiais dos personagens, sendo sempre determinadas pelo tipo de aventura. Estas habilidades podem ser específicas ou genéricas, sendo que neste caso é preciso concordância de todos os jogadores da mesa.

Ao narrar uma aventura, o Mestre precisa determinar quais perícias podem existir naquele universo ficcional ainda antes da criação dos personagens. Isso é necessário para evitar situações como o personagem está em uma aventura no Velho Oeste, mas "comprou" a perícia \emph{Hacker de computadores}. Em outras palavras, é preciso que as perícias dos jogadores façam sentido no universo em que a aventura acontece. 

Recomenda-se que o Mestre faça uma lista com 15 a 20 perícias compatíveis com o tipo de aventura que será narrado, para a criação dos personagens. Neste momento os jogadores poderão "comprar" as perícias escolhidas com os pontos de perícia. 

O Sistema +2d6@ifsul (bem como o Sistema +2d6 original) permitem a criação de perícias personalizadas, entretanto, este Livro de Regras não discute este mecanismo. Para saber mais sobre este assunto o leitor poderá acessar as regras originais do Sistema +2d6\footnote{\url{https://newtonrocha.wordpress.com/sistema-de-rpg-2d6/}}.  

\section{\label{sec:usoPericia} Comprando e Usando perícias}
Durante o jogo, a perícia é usada de forma complementar a algum atributo, ou seja, é importante "alinhar" perícias com os atributos registrados na ficha. Desta forma, a descrição do personagem fica coerente com a pontuação na ficha.

Se eu tenho um personagem que possui mais força se comparado aos outros atributos é lógico (mas não obrigatório) que eu procure comprar perícias relacionadas a força.

\begin{center}
	====================================================
\end{center}

Ao lado do nome de cada perícia está o Atributo Básico, ao qual ela deve ser somada em um teste de ação ou oposto (ver mais na seção Jogadas de testes). 

Para efeitos de jogo, as perícias são classificadas em valores numéricos, da seguinte forma: (1) Iniciante, (2) Profissional, (3) Experiente, (4) Perito, (5) Expert.

As perícias apresentadas nas seções a seguir são sugestões comuns. Os jogadores de uma mesa podem criar novas perícias desde que todos os jogadores estejam de comum acordo.

Por exemplo: Na listagem seção~\ref{sec:perFisicos} existe a \emph{Resistência à venenos}. Mestre e jogadores poderiam criar uma nova perícia "semelhante" - \emph{Resistência ao frio}.
\section{\label{sec:pericias}Lista de perícias mais comuns}


\subsection{\label{sec:perArmas}Perícias com armas}
\begin{itemize}
	\item Armas comuns (FOR ou DES): Você sabe utilizar armas comuns como facas, machadinhas, porretes, facão machete, funda, besta de mão, arco curto, etc.
	\item Armas profissionais corpo-a-corpo (FOR): Você sabe usar armas profissionais de combate corpo-a-corpo, como espadas de todos os tipos 	
	\item Armas profissionais à distância (DES): Você sabe usar armas profissionais à distância como arcos, arma de fogo, arma de energia, lanças, etc.
\end{itemize}

\subsection{\label{sec:perFisicos}Perícias baseadas de atributos físicos}
Perícias baseadas em atributos físicos (DES, FOR ou CON):
\begin{itemize}
	\item Acrobacia (DES): Você sabe executar saltos mortais ou acrobáticos, estrelas, salto em distância ou altura, equilibrismo, etc.
	\item Arte da Fuga (DES): Você possui habilidades para escapar de amarras, correntes ou algemas, rastejar por espaços apertados,etc.
	\item Atletismo (FOR): Você possui treinamento de força e consegue executar tarefas que exijam força muscular como levantar objetos pesados, carregar grandes cargas, etc.
	\item Arremessar (DES): Você sabe lançar pequenos objetos usando apenas as mãos (pedras, dardos, etc).
	\item Artes Marciais (DES): Você conhece artes marciais. Dependendo do tipo de aventura, pode-se criar uma perícia específica para cada tipo de arte marcial (Judô, Karatê, Jiu-jitsu, etc).
	\item Cavalgar (DES): Você sabe andar a cavalo (ou outro animal de carga equivalente).
	\item Controle da Respiração (CON): Você conhece técnicas para prender ou controlar a respiração por 1d6 minutos.
	\item Escalar (FOR ou DES): Você conhece técnicas para escalar superfícies íngremes a uma velocidade igual ao Deslocamento/4.
	\item Esconder-se (DES): Você tem habilidade de esconder-se com facilidade.
	\item Ferreiro (DES): Você sabe trabalhar com forjas e moldar o metal.
	\item Furtividade (DES): Você tem a habilidade de aproximar-se sem ser notado, andar silenciosamente, passar despercebido, etc.
	\item Natação (FOR ou DES): Você sabe nadar e mergulhar.
	\item Punga (DES): Você desenvolveu habilidades para roubar pessoas sem ser percebido (roubar carteira, celular, documentos).
	\item Resistência a venenos (CON): Você é resistente à venenos.
	\item Tolerância (CON): Possui a capacidade de resistir dor, fadiga, fome e sede.
\end{itemize}

\section{\label{sec:perMentais}Perícias de atributos mentais}

\begin{itemize}
	\item Pilotar Veículos Terrestres (INT): Você sabe pilotar veículos terrestres como carros, caminhões, motocicletas, etc.
	\item Pilotar Veículos Aquáticos (INT): Você sabe pilotar veículos aquáticos como barcos, embarcações, jangadas, etc.
	\item Pilotar Veículos Aéreos (INT): Você sabe pilotar veículos aéreos como pequenos aviões, helicópteros, balões a gás, asa-delta, etc.
\end{itemize}


Administração (INT): conhecimento de gerência, liderança, coisas corporativas, etc.
Alquimia (INT): conhecimento alquímico, preparo de poções mágicas, talismãs, etc.
Antropologia (INT): estudo do homem e da cultura humana, estudo de culturas primitivas e contemporâneas.
Arrombamento (INT): saber abrir fechaduras, abrir portas, cofres, etc.
Artilharia (INT): conhecimento de artilharia militar, artilharia antiaérea, etc.
Biologia (INT): conhecimento acadêmico sobre os seres vivos.
Camuflagem (INT): saber como se camuflar, esconder, criar disfarces, etc.
Cartografia (INT): habilidade de criar mapas, de mapear locais desconhecidos, etc.
Ciência Forense (INT): conhecimento técnico e acadêmico sobre causas dos crimes, causas de mortes, autópsias, etc.
Ciências Proibidas/Ciências Ocultas (INT): ocultismo, arcanismo e outras ciências não oficiais, que envolvem temas tabus e proibidos como magia, demonologia, estudo dos espíritos, etc.
Comércio (INT): perícia genérica que envolve barganha, negociação,reconhecer o melhor preço, etc.
Computação (INT): conhecimento acadêmico ou técnico sobre computadores, análise de sistemas, programação, reparo.
Conhecimento <área> (INT): perícia genérica, pode ser usada junto com alguma área de conhecimento, para RPGs que não sejam muito investigativos, como Conhecimento Ciências Humanas, Técnico, Eletrônico, etc.
Criptografia (INT): conhecimento de como quebrar ou criar códigos secretos.
Cultura Popular (INT): conhecimento dos rumores, boatos, acontecimentos do mundo do entretenimento, cultura de massa, novelas, celebridades, etc.
Direito (INT): conhecimento da legislação, conhecimento dos procedimentos legais, de como se defender e atacar em um tribunal, reconhecer ilegalidades.
Disfarces (INT ou CAR): habilidade de criar disfarces, de se passar por outra pessoa.
Economia (INT): conhecimentos de economia, finanças nacionais, problemas econômicos.
Eletrônica (INT): conhecimento de sistemas eletrônicos, criação, projeto, hacking, etc.
Ensino (INT): conhecimento didático, habilidade de dar aulas, de passar informações.
Espionagem (INT): perícia genérica, conjunto de habilidades relacionadas com espiões, como instalação de câmeras secretas, perseguir sem ser visto, desarmar esquemas de segurança,etc.
Escrever (INT): habilidade de escrever profissionalmente.
Estratégia Militar (INT): conhecimento de táticas militares, capaz de reconhecer táticas militares do inimigo, descobrir pontos fracos, estabelecer táticas de combate eficazes.
Explosivos (INT): armar e desarmar explosivos, criar bombas, etc.
Exorcismo (INT ou CAR): sabe realizar e conhece rituais de exorcismo, etc.
Falsificação (INT): habilidade de falsificar documentos, objetos de arte, dinheiro, etc.
Finanças (INT): perícia genérica que envolve economia, contabilidade, áreas que envolvem dinheiro como investimentos, taxas, etc.
Geologia (INT): estudo do solo, eras geológicas, etc.
Hacking (INT): habilidade para invadir sistemas, criar vírus de computador, tomar controle de redes de telecomunicações, etc.
Hipnose (INT): habilidade para afetar a mente de uma pessoa e torná-la mais fácil de ser manipulada.
História (INT): estudo da história da humanidade, conhecimento de fatos históricos locais ou globais.
Investigação (INT ou SAB): perícia genérica, habilidade técnica de reunir pistas, reunir evidências, deduzir, etc.
Literatura (INT): conhecimento acadêmico de obras literárias, capacidade de interpretação profissional de obras de arte, conhecimento de história da literatura.
Magia (INT/POD): tem conhecimento de magias arcanas.
Magia (SAB/POD): tem conhecimento de magias divinas. 
Matemática (INT): estudo da ciência da matemática.
Mecânica (INT): habilidade de criar e reparar equipamentos mecânicos, consertar veículos.
Medicina (INT): conhecimento de medicina, realizar operações, tratar ferimentos, etc.
Mitos de Cthulhu (INT): conhecimento dos inomináveis horrores cósmicos.
Naturalista (INT): conhecimento acadêmico sobre a vida selvagem.
Navegação (INT): sabe conduzir embarcações e se orientar em alto mar, etc.
Ofícios <área> (INT): perícia genérica para representar algum tipo de habilidade técnica como artesanato, escultura, ferraria, criação de animais, agricultura, joalheria, etc.
Operar Submarino/Navio/Nave Espacial (INT): perícias específicas sobre como cuidar, operar, pilotar, o veículo indicado.
Procurar (SAB): habilidade de vasculhar com atenção, notar pequenas diferenças, muito útil para investigadores.
Psicologia e psiquiatria (INT): conhecimento acadêmico de psicologia, e saber realizar terapias, hipnotizar, analisar e entender as motivações ocultas das ações dos indivíduos.
Química (INT): conhecimento de química, reações e elementos químicos, etc.
Rastreio (INT): saber rastrear, seguir rastros, perseguir uma presa ou pessoa.
Religião (INT ou SAB): conhecimento teológico, de história das religiões, da ritualística, etc.
Rituais Ocultistas (INT ou POD): conhecimento de como realizar rituais mágicos.
Rituais Religiosos (INT): conhecimento de como realizar rituais religiosos.
Senso de Direção (SAB): habilidade para distinguir as direções através do instinto.
Sobrevivência (INT): habilidade de sobreviver em lugares selvagens, encontrar abrigo, caçar, conseguir alimento, reconhecer plantas venenosas.
Venenos (INT): saber criar, usar, identificar e neutralizar venenos.
Veterinário (INT): tratar ferimentos e doenças de animais, conhecimento de biologia animal.
Vontade (SAB): o personagem tem a Força de Vontade treinada, capaz de resistir ao medo, à hipnose, poderes mentais, etc.

