\chapter[Apresentação]{\label{ch:apres}Apresentação}
%\addcontentsline{toc}{chapter}{Apresentação}

Este livro de regras surgiu após a finalização do projeto de extensão “RPG na Biblioteca Pública: Potencializando Talentos”, executado no segundo semestre de 2023, com apoio financeiro do Instituto Federal Sul-rio-grandense, pelo edital geral de fomento Nº 02/2023- PROEX. 

Na ocasião, a equipe envolvida buscava um sistema de RPG (sob licença livre/aberta), que pudesse ser adaptado às necessidades do projeto. Vários sistemas foram testados e avaliados, sendo que o sistema selecionado foi o +2d6, criado pelo prof. Newton Rocha (Tio Nitro)\footnote{\url{https://newtonrocha.wordpress.com/}}. 

Ainda em 2023, fizemos contato com o prof. Newton, que prontamente autorizou a adaptação, originando assim o \emph{Sistema +2d6\_IFSul} (Versão 1.0). Em 2024 este livro de regras é organizado, contendo a essa primeira versão revisada e renomeada para \emph{Sistema +2d6@IM}. A principal adaptação é relacionada aos atributos do personagem, que passa utilizar conceitos das \emph{Inteligências Múltiplas}, de Howard Gardner\footnote{The Theory of Multiples Intelligences (1987).}.

\textcolor{red}{Segundo a Teoria das Inteligências Múltiplas (Gardner, H. 1987) uma pessoa possui várias inteligências, que podem ser vistos como potencias humanos. }

O esforço para concretização, divulgação e utilização deste material, é nutrido por evidências e também pela (forte) crença, de que o RPG de Mesa possui grande potencial para o estímulo da leitura, do trabalho colaborativo e desenvolvimento da imaginação e criatividade.

\section{\label{sec:cred}Licenciamento} 
Esta seção apresenta os créditos e a licença de uso deste trabalho.

\subsection*{Atribuição de créditos}
Este trabalho é uma adaptação do \emph{Sistema +2d6}, desenvolvido por \href{http://newtonrocha.wordpress.com/}{Newton "Tio Nitro" Rocha}, originalmente licenciado com Creative Commons 3.0 \ccby.

\subsection*{Licenças paro o Sistema +2d6@IM}
\href{https://github.com/josefigueiredo/Sistema2d6-ifsul}{Sistema +2d6@IM}\footnote{Para efeitos de licenciamento, o \emph{Livro de Regras} e o \emph{Sistema +2d6@IM} são considerados o mesmo objeto, denominado apenas \emph{Sistema +2d6@IM}.} © 2024 by \href{https://josefigueiredo.github.io/}{José Antônio de Figueiredo} is licensed under \href{https://creativecommons.org/licenses/by-sa/4.0/?ref=chooser-v1}{Creative Commons Attribution-ShareAlike 4.0 International} \ccbysa.

Nesta licença você \textbf{tem o direito de:}
\begin{itemize}
	\item Compartilhar — copiar e redistribuir o material em qualquer suporte ou formato para qualquer fim, mesmo que comercial.
	\item Adaptar — remixar, transformar, e criar a partir do material para qualquer fim, mesmo que comercial.
\end{itemize}

O licenciante não pode revogar estes direitos desde que você respeite os termos da licença.

\textbf{De acordo com os termos seguintes:}\\
\begin{itemize}
	\item \ccAttribution \textbf{ Atribuição} — Você deve dar o crédito apropriado , prover um link para a licença e indicar se mudanças foram feitas. Você deve fazê-lo em qualquer circunstância razoável, mas de nenhuma maneira que sugira que o licenciante apoia você ou o seu uso.
	\item \ccShare \textbf{ CompartilhaIgual} — Se você remixar, transformar, ou criar a partir do material, tem de distribuir as suas contribuições sob a mesma licença que o original.
	\item \textbf{Sem restrições adicionais} — Você não pode aplicar termos jurídicos ou medidas de caráter tecnológico que restrinjam legalmente outros de fazerem algo que a licença permita.	
\end{itemize}

\textbf{Avisos:}
\begin{itemize}
	\item Você não tem de cumprir com os termos da licença relativamente a elementos do material que estejam no domínio público ou cuja utilização seja permitida por uma exceção ou limitação que seja aplicável.
	\item Não são dadas quaisquer garantias. A licença pode não lhe dar todas as autorizações necessárias para o uso pretendido. Por exemplo, outros direitos, tais como direitos de imagem, de privacidade ou direitos morais , podem limitar o uso do material.
\end{itemize}
