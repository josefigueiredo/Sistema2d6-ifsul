\chapter[Apresentação]{\label{ch:apres}Apresentação}
%\addcontentsline{toc}{chapter}{Apresentação}

Este livro de regras surgiu após a finalização do projeto de extensão “RPG na Biblioteca Pública: Potencializando Talentos”, executado no segundo semestre de 2023. O projeto teve apoio financeiro do Instituto Federal Sul-rio-grandense, pelo edital geral de fomento Nº 02/2023- PROEX. 

Na ocasião, a equipe envolvida buscava um sistema de RPG (sob licença livre/aberta), que pudesse ser adaptado às necessidades do projeto. Vários sistemas foram testados e avaliados, sendo que o sistema selecionado foi o +2d6, criado pelo prof. Newton Rocha (Tio Nitro)\footnote{\url{https://newtonrocha.wordpress.com/}}. 

Ainda em 2023, fizemos contato com o prof. Newton que prontamente autorizou a adaptação, originando assim o Sistema +2d6\_IFSul (Versão 1.0). Em 2024 esta primeira versão é revisada, sendo então renomeado para +2d6@ifsul e originando este livro de regras.

O RPG de Mesa é um tipo de jogo onde os participantes controlam um personagem dentro de uma história narrada. Um destes jogadores, normalmente chamado Mestre, é responsável por conduzir/narrar a história, garantindo que os participantes vivenciem dos eventos ocorridos. 

Utilizando a imaginação e criatividade, os jogadores conduzem seus personagens interagindo com outros personagens, com o ambiente, com mecanismos, armadilhas, tesouros e tudo o mais que possa existir em uma boa história. Estando inserido no universo ficcional, os personagens também estão sujeitos aos grandes eventos desse mundo como cataclismos, invasões alienígenas, profecias, maldições , lendas, etc. Estes mesmos personagens, pela sua ação, também podem causar mudanças neste mundo ficcional.

O jogo é colaborativo, ou seja, os participantes não jogam um contra o outro, não existindo um vencedor. Todos precisam colaborar entre si para obterem sucesso na resolução do(s) conflito(s) da narrativa. Em teoria não há limites para a exploração destes aspectos e, em função disso, é lógico inferir que o RPG proporciona um forte estímulo para o desenvolvimento de habilidades humanas como imaginação e criatividade, habilidade de trabalho em equipe, planejamento na solução de problemas, entre outros.

Além disso, ao adentrar no universo do RPG de Mesa, naturalmente os jogadores desenvolvem (ou aumentam) o hábito da leitura, mesmo que de forma não convencional. A imersão no RPG impele o jogador na busca pela compreensão dos sistemas de regras, das histórias que estão por trás de cada personagem ou cenário, e, principalmente, na busca de elementos para enriquecer a própria criatividade.

Assim, o esforço para concretização e utilização deste material, origina-se na forte crença de que o RPG de Mesa tem grande potencial para o estímulo da leitura, do trabalho colaborativo e do desenvolvimento da imaginação e criatividade.

\section{\label{sec:cred}Licenciamento} 
Esta seção apresenta os créditos e a licença de uso deste trabalho.

\noindent \textbf{Atribuição de créditos:} \\
\indent Este trabalho é uma adaptação do \emph{Sistema +2d6}, desenvolvido por \href{http://newtonrocha.wordpress.com/}{Newton "Tio Nitro" Rocha}, originalmente licenciado com Creative Commons 3.0 \ccby.

\noindent \textbf{Licenças paro o Sistema +2d6@ifsul:} \\
\indent \href{https://github.com/josefigueiredo/Sistema2d6-ifsul}{Sistema +2d6@ifsul}\footnote{Para efeitos de licenciamento, o \emph{Livro de Regras} e o \emph{Sistema +2d6@ifsul} são considerados o mesmo objeto, denominado apenas \emph{Sistema +2d6@ifsul}.} © 2024 by \href{https://josefigueiredo.github.io/}{José Antônio de Figueiredo} is licensed under \href{https://creativecommons.org/licenses/by-sa/4.0/?ref=chooser-v1}{Creative Commons Attribution-ShareAlike 4.0 International} \ccbysa.

\textbf{Você tem o direito de:}
\begin{itemize}
	\item Compartilhar — copiar e redistribuir o material em qualquer suporte ou formato para qualquer fim, mesmo que comercial.
	\item Adaptar — remixar, transformar, e criar a partir do material para qualquer fim, mesmo que comercial.
\end{itemize}

O licenciante não pode revogar estes direitos desde que você respeite os termos da licença.

\textbf{De acordo com os termos seguintes:}\\
\begin{itemize}
	\item \ccAttribution \textbf{ Atribuição} — Você deve dar o crédito apropriado , prover um link para a licença e indicar se mudanças foram feitas. Você deve fazê-lo em qualquer circunstância razoável, mas de nenhuma maneira que sugira que o licenciante apoia você ou o seu uso.
	\item \ccShare \textbf{ CompartilhaIgual} — Se você remixar, transformar, ou criar a partir do material, tem de distribuir as suas contribuições sob a mesma licença que o original.
	\item \textbf{Sem restrições adicionais} — Você não pode aplicar termos jurídicos ou medidas de caráter tecnológico que restrinjam legalmente outros de fazerem algo que a licença permita.	
\end{itemize}

\textbf{Avisos:}
\begin{itemize}
	\item Você não tem de cumprir com os termos da licença relativamente a elementos do material que estejam no domínio público ou cuja utilização seja permitida por uma exceção ou limitação que seja aplicável.
	\item Não são dadas quaisquer garantias. A licença pode não lhe dar todas as autorizações necessárias para o uso pretendido. Por exemplo, outros direitos, tais como direitos de imagem, de privacidade ou direitos morais , podem limitar o uso do material.
\end{itemize}
